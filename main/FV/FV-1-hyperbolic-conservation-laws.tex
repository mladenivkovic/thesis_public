%----------------------------------------
\chapter{Hyperbolic Conservation Laws}
%----------------------------------------

First, let's get a little more concrete.
Formally, conservation laws are systems of partial differential equations (PDEs) that can be
written in the form
%
\begin{align}
  \deldt \U + \deldx \F(\U) &= 0 \\[.5em] \label{eq:conservation-law-1D-introduction}
  \text{with } \U = \left( \begin{matrix}
                    \uc_1 \\ \uc_2 \\ \vdots \\ \uc_m \\
                    \end{matrix} \right), & \quad \quad
                \F =\left( \begin{matrix}
                    \fc_1 \\ \fc_2 \\ \vdots \\ \fc_m \\
                    \end{matrix} \right)
\end{align}
%
where $\U$ is called the vector of conserved states, and $\F$ is the vector of fluxes, and in
general is a function of the state vector. This definition of conservation laws is valid for one
spatial dimension. However, the extension to multiple dimensions is straightforward: Instead of
being a vector, the flux becomes a tensor, and the partial spatial derivative is a divergence
operator instead.

A PDE of the form
%
\begin{align}
    \deldt &\U + \mathcal{A} \deldx \U + \mathcal{B} = 0 \\[.5em]
  \text{with }  &\U = \left( \begin{matrix}
                    u_1 \\ u_2 \\ \vdots \\ u_m \\
                    \end{matrix} \right),  \quad \quad
                \mathcal{B} = \left( \begin{matrix}
                    b_1 \\ b_2 \\ \vdots \\ b_m \\
                    \end{matrix} \right), & \quad \quad
                \mathcal{A} =\left( \begin{matrix}
                    a_{11} & \ldots & a_{1m} \\
                    \vdots & \ddots & \vdots \\
                    a_{m1} & \ldots & a_{mm} \\
                    \end{matrix} \right)
\end{align}
%
%
is called

\begin{itemize}
 \item linear with constant coefficients if $a_{ij} = \CONST$ and $b_i = \CONST$
 \item linear with variable coefficients if $a_{ij} = a_{ij}(x, t)$ and $b_i = b_i(x, t)$ or
$\mathcal{B} = \alpha + \beta \U$
 \item  quasi-linear if $\mathcal{A} = \mathcal{A}(\U)$
 \item homogeneous if $\mathcal{B} = 0$
\end{itemize}


A conservation laws like \ref{eq:conservation-law-1D-introduction} can be written in homogeneous
quasi-linear form:

\begin{align}
 \DELDT{\U} + \DELDX{F} = 0 = \DELDT{\U} + \frac{\del \F}{\del \U} \DELDX{U} = \DELDT{U} +
\mathcal{A}(\U) \DELDX{U}
\end{align}

where $\mathcal{A}$ is the Jacobian matrix of the flux function $\F(\U)$ and has the elements
$a_{ij} = \frac{\del f_i}{\del u_j}$.

Furthermore, a conservation law \ref{eq:conservation-law-1D-introduction} is said to be hyperbolic
at a point $(x, t)$ if $\mathcal{A}$ has $m$ real eigenvalues $\lambda_1$, ..., $\lambda_m$ and a
corresponding set of $m$ linearly independent right eigenvectors $\mathbf{K}^{(1)}$, ...,
$\mathbf{K}^{(m)}$. If the $\lambda_i$ are all distinct, the system is called strictly hyperbolic.
The fact that the conservation laws under consideration are hyperbolic, i.e. have $m$ eigenvalues
and linearly independent eigenvectors, is a crucial element for the construction of the numerical
solution strategy using finite volume methods.




As mentioned before, this part of the work will focus on the linear advection equation and the
Euler equation which govern ideal gases. The linear advection equation in its general,
time-dependent form in three dimensions is given by:

\begin{align}
  \deldt{\uc} +
    a(\x, t) \deldx{\uc} +
    b(\x, t) \frac{\partial}{\partial y} \uc +
    c(\x, t) \frac{\partial}{\partial z} \uc
    = 0 \ .
    \label{eq:linear-advection-general}
\end{align}

If the coefficiens are sufficiently smooth (i.e. differentiable), it can be expressed as a
conservation law with source terms:

\begin{align}
  \deldt{\uc} +
    \deldx{(a \uc)} +
    \frac{\partial}{\partial y} (b\uc) +
    \frac{\partial}{\partial z} (c \uc )
    = \uc \left(
    \deldx{a} +
    \frac{\partial}{\partial y} b +
    \frac{\partial}{\partial z} c
    \right) \ .
\end{align}

In its simplest form, which is also the form that will be made heavy use of in this work, we only
consider one dimension and a constant coefficient $a$:

\begin{align}
    \deldt{\uc} + a \deldx{\uc} = 0 \label{eq:linear-advection-1D-const-coeff}
\end{align}



The Euler equations are given by:


\begin{align}
	\deldt
	\begin{pmatrix}
		\rho \\
		\rho \V \\
		E
	\end{pmatrix}
	+
	\nabla \cdot
	\begin{pmatrix}
		\rho \V\\
		\rho \V \otimes \V + p\\
		(E + p) \V
	\end{pmatrix}
	=
	\begin{pmatrix}
		0\\
		\rho \mathbf{a}\\
		\rho \mathbf{a} \V
	\end{pmatrix}
	\label{eq:euler-equations}
\end{align}



where
\begin{itemize}
	\item $\rho$ is the fluid density
	\item $\V = (v_1, v_2, v_3)^T$ is the fluid velocity at a given point. This is the mean
or bulk velocity of the fluid, not the velocity of individual particles that compose the fluid.
	\item $p$ is the pressure
	\item $E$ is the specific energy. $E = \frac{1}{2} \rho \V^2 + \rho u$, with $u$ being the
specific internal thermal energy.
	\item $\mathbf{a}$ is an acceleration due to some external force.
\end{itemize}

The outer product $\cdot \otimes \cdot$ gives the following tensor:
\begin{align}
	(\V \otimes \V)_{ij} = v_i v_j
\end{align}


Furthermore, for ideal gases, we have the equation of state:
%
\begin{align}
	p &= n k_B T && \text{ideal gas law} \\
\end{align}
%
which can also be written as:
%
\begin{align}
	u &= \frac{1}{\gamma - 1}\frac{p}{\rho} \label{eq:equation-of-state}
\end{align}
%
and the following relations:
%
\begin{align}
	c_s &= \sqrt{\frac{\del p}{\del \rho} \bigg{|}_s } = \sqrt{\frac{\gamma p}{\rho}} &&
\text{sound speed of the gas} \\
	s &= c_V \ln \left( \frac{p}{\rho^\gamma} \right) + s_0 && \text{entropy} \label{eq:entropy}\\
	p &= C \rho ^ \gamma && \text{entropy relation for smooth flow, i.e. no shocks}
\end{align}

where
\begin{itemize}
	\item $n$ is the number density of the gas
	\item $k_B$ is the Boltzmann constant
	\item $T$ is the temperature
	\item $s$ is the entropy
	\item $\gamma$ is the adiabatic index
    \item $c_V$ is the specific heat capacity at constant volume
\end{itemize}


In 1D, we can write the Euler equations without source terms ($\mathbf{a} = 0$) and dropping the
index $1$ from the velocity term $v_1$ as:

\begin{align}
	\deldt{
		\begin{pmatrix}
			\rho \\ \rho v \\ E
		\end{pmatrix}
		}
	+ \deldx {
		\begin{pmatrix}
			\rho v\\
			\rho v^2  + p\\
			(E + p) v
		\end{pmatrix}
	} = 0 \ .
	\label{eq:euler-equations-1D}
\end{align}
