%-----------------------------
\chapter{Introduction}
%-----------------------------

Conservation laws take a prominent role in physics. They pop up everywhere where some conserved
quantity evolves dynamically, i.e. has fluxes, and allow us to establish a formal description of
physical systems. Conserved quantities themselves are beloved for the same reason, and it helps that
there is no scarcity of them. In fact, the Noether theorem
\citep{noetherInvarianteVariationsprobleme1918} states that for each symmetry of action of a system,
a conserved quantity (and a corresponding conservation law) can be found. For example, time
invariant systems conserve energy, translationally invariant systems conserve momentum, and
rotationally invariant systems conserve angular momentum. Some concrete examples of conservation
laws are the charge conservation in electrical currents, the Jeans equations, the heat flow in an
uniform rod, the traffic flow equation, and the shallow water equations. In this part of the
thesis, I will focus mainly on two sets of conservation laws: The first are the equations of linear advection, whose simple form allows to find exact solutions for many problems arising in numerical
methods which solve conservation laws, such as the finite volume methods. The second set that will
be a major focus in this Part of this work are the Euler equations, which describe the mass,
momentum, and energy conservation of an ideal fluid. While the linear advection equation in this
Part will mostly be used as a model and example for a conservation law, the solution of Euler
equations constitutes an essential building block for simulations of astrophysical systems. Details
of the linear advection equation and the Euler equations are discussed in the subsequent sections.
Finally, in Part \ref{part:rt} the moments of the equations of radiative transfer, which also take
form of a hyperbolic conservation law, are discussed and solved. This part of the work aims to
establish a rudimentary overview and understanding of how finite volume methods in the context of
hyperbolic conservation laws work, and what complexities and limitations they contain. The outline
follows selected topics from \citet{toroRiemannSolversNumerical2009} and
\citet{levequeFiniteVolumeMethods2002}. The content won't be reserved to only theory and equations
though - where applicable, actual results from simulations using the \meshhydro code are
shown and discussed. \meshhydro is a code that I have developed in scope of this work to
serve as a simple didactical tool. It was created to solve linear advection as well as the Euler
equations using various Riemann solvers, slope and flux limiters, integration schemes, and even
external source terms. Furthermore, it is extensively documented, it contains a range of
ready-to-run examples, and comes with a suite of visualization and analysis tools written in Python
3. It is however limited to discretizing space in a regular grid only, and only in one or two
dimensions. It is also not parallelized at all. \meshhydro is open source software and
available on \url{https://github.com/mladenivkovic/mesh-hydro}.
