%==================================
\chapter{Introduction}
%==================================

Mock galaxy catalogues generated using N-body or hydrodynamical
simulations are important tools for extragalactic astronomy and
cosmology.  They are used to test current theories of galaxy
formation, to explore systematic and statistical errors in large scale
galaxy surveys and to prepare analysis codes for future dark energy
missions such as Euclid or LSST.  There is a large variety of methods
to generate such mock galaxy catalogues.  The most ambitious line of
products is based on full hydrodynamical simulations, where dark
matter, gas, and star formation are directly simulated
\citep[e.g.][]{duboisDancingDarkGalactic2014,
  khandaiMassiveBlackIISimulationEvolution2015,
  vogelsbergerPropertiesGalaxiesReproduced2014,
  schayeEAGLEProjectSimulating2015}.
The intermediate approach is based on semi-analytic modelling
(hereafter SAM) \citep[e.g.][]{whiteGalaxyFormationHierarchical1991,
  bowerBreakingHierarchyGalaxy2006, somervilleSemianalyticModellingGalaxy1999,
    kauffmannFormationEvolutionGalaxies1993,
  kangSemianalyticalModelGalaxy2005,crotonManyLivesActive2006} for
which galaxy formation physics, although simplified, is still at the
origin of the mock galaxy properties. Finally, the simplest and most
flexible approach is based on a purely empirical modelling of galaxy
properties, sometimes called Halo Occupation Density (HOD hereafter)
\citep[e.g.][]{seljakAnalyticModelGalaxy2000, berlindHaloOccupationDistribution2002,
  peacockHaloOccupationNumbers2000,
  bensonNatureGalaxyBias2000,wechslerGalaxyFormationConstraints2001,
  scoccimarroHowManyGalaxies2001}.
The last two techniques (SAM and HOD) both require the complete
formation history of dark matter
haloes, and possibly their sub-haloes. This formation history is
described by halo `\emph{merger trees}'
\citep{roukemaSpectralEvolutionMerging1993,
  roukemaFailureSimpleMerging1993, laceyMergerRatesHierarchical1993}.
Accurate merger trees are essential to obtain realistic mock galaxy
catalogues, and constitute the backbone of SAM and HOD models.


The advantage of using SAM and HOD techniques to generate mock galaxy
catalogues is that one does not need to model explicitly the gas
component, but only the dark matter component.  The corresponding
N-body simulations are commonly referred to as `\emph{dark matter
only}' (DMO) simulations.  With growing processing power, improved
algorithms and the use of parallel computing tools and architectures,
larger and better resolved DMO simulations are becoming possible.  The
current state-of-the-art is the Flagship simulation performed for the
preparation of the Euclid mission \citep{potterPKDGRAV3} and featured 2
trillion dark matter particles.  Such extreme simulations make
post-processing analysis tool such as merger tree algorithms
increasingly difficult to develop and to use, mostly because of the
sheer size of the data to store on disk and to load up later from the
same disk back into the processing unit memory.  In some extreme
cases, the amount of data that needs to be stored to perform a merger
tree analysis in post-processing is simply too large.  Storing just
particle positions and velocities in single precision for trillions of
particles requires dozens of terabytes per snapshot.  Another issue is
that most modern astrophysical simulations are executed on large
supercomputers which offer large distributed memory.  Post-processing
the data they produce may also require just as much memory, so that
the analysis will also have to be executed on the distributed memory
infrastructures as well.  The reading and writing of such vast amount
of data to a permanent storage remains a considerable bottleneck,
particularly so if the data needs to be read and written multiple
times.  One way to reduce the computational cost is to include
analysis tools like halo-finding and the generation of merger trees in
the simulations and run them ``\textit{on the fly}'', i.e. run them
during the simulation, while the necessary data is already in memory.

The main motivation for this work is precisely the necessity for such
a merger tree tool for future ``beyond trillion particle''
simulations. While many state-of-the-art N-body simulation codes include
structure finders that are run on-the-fly, codes like \codename{Gadget4}
\cite{springelSimulatingCosmicStructure2021} who are able to build
merger trees on-the-fly are still a relative rarity. It is crucial that
multiple, distinct, codes have the capacity to do this to provide the
possibility to cross-check results and their convergence.
To this end, a new algorithm that we named
\acacia was designed to work on the fly within the parallel
AMR code \ramses.  One novel aspect of this work is the use of the
halo finder \phew\ \citep{bleulerPHEWParallelSegmentation2015} for the parent halo catalogue.
Different halo finders have been shown to have a strong impact on the
quality of the resulting merger trees \citep{SUSSING_HALOFINDER}.
\phew\ falls into the category of ``watershed'' algorithms that are
not so common in the cosmological halo finding literature.  This type
of algorithm assigns particles (or grid cells) to density peaks above
a prescribed density threshold and according to the so-called
``watershed segmentation'' of the negative density field.

This Part of the work is a slightly modified version of the corresponding published article
\citep{ivkovicACACIANewMethod2022} to fit the format of a thesis better. While the bulk of the
algorithm development and implementation precedes the starting point of my thesis, a substantial
effort during my thesis was directed towards finishing the project, fixing a handful of bugs,
extending the performance analysis of the algorithm, and validating the quality of the resulting
merger trees by generating mock galaxy catalogues and comparing them to observational data.

This Part is structured as follows. In Chapter \ref{chap:phew}, a brief description of the \phew\
halo finder and its new particle unbinding method is given. Chapter \ref{chap:making_trees}
describes some common difficulties that arise when making merger trees and the way we address them
in our merger algorithm \acacia, which is ultimately described in detail in Chapter
\ref{chap:my_code}. Chapter \ref{chap:tests} shows test results to determine what parameters give
the best results.  Using the halo catalogue and its corresponding merger tree generated on the fly
by a cosmological N-body simulation, we use the stellar-mass-to-halo-mass (SMHM) relation from
\cite{behrooziAVERAGESTARFORMATION2013} to produce a mock galaxy catalogue. We analyse in
Section~\ref{chap:mock_catalogues} the properties of our mock galaxy catalogue and show that the
introduction of orphan galaxies improve the comparison to observations considerably.

The \ramses\ code is  publicly available and  can be  downloaded from
\url{https://bitbucket.org/rteyssie/ramses/}. Instructions  on how to use \acacia\ and \phew\ during
a simulation can be found under \url{https://bitbucket.org/rteyssie/ramses/wiki/Content}.
