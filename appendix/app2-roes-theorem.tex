%====================================================================================
\chapter{Roe's Theorems On The Accuracy Of A Method For Scalar Conservation Laws}
\label{app:roe}
%====================================================================================

In this Appendix, Roe's theorems on the accuracy of a numerical method for the linear advection
equation are derived. The original reference for these theorems,
\cite{roep.l.NumericalAlgorithmsLinear1981}, is unfortunately unpublished, so a more practical
reference would be \cite{billetAccuracyStabilityExplicit1997}, who have derived the theorems for
two and three dimensional problems as well. Here, we only look at the one dimensional case.

Consider the linear advection equation

\begin{align}
 \deldt \uc + a \deldx \uc = 0 \label{eq:app-roe-linear-advection}
\end{align}

for a constant wave speed $a$. Consider a general scheme to solve the linear advection equation

\begin{align}
 \uc_i^{n+1} = \sum_\alpha A_\alpha u_{i+\alpha} ^ n \label{eq:app-roe-general-scheme}
\end{align}

where $\uc_i^n = \uc (i \Delta x, n \Delta t)$, and $\{A_\alpha \}$ is a finite set of constant
nonzero coefficients. $\alpha$ denotes the indices of the stencil of the method. For example in
Godunov's method (eq.~\ref{eq:godunov-advection}), the update formula for $\uc_i^{n+1}$ depends on
the states $\uc_{i-1}$ and $\uc_{i}$, and hence for that example $\alpha \in \{-1, 0 \}$.


The first theorem is as follows: If $\uc_i^n$ is a polynomial of degree $p$ in $i$,
scheme~\ref{eq:app-roe-general-scheme} will give the exact solution
to~\ref{eq:app-roe-linear-advection} if and only if

\begin{align}
 \sum_\alpha A_\alpha \alpha^q = (- C_{CFL})^q \label{eq:app-roe-theorem1}
\end{align}

for all integers $0 \leq q \leq p$, where $C_{CFL} = \frac{a \Delta t}{\Delta x}$.

To prove this theorem, we take some initial $\uc_i^{n} = \uc(i \Delta x, n \Delta t)$ to be a
polynomial of degree $p$:

\begin{align}
    \uc_i^{n} = \sum_{q = 0}^p \beta_q (i \Delta x)^q
\end{align}

where $\beta_q$ are some constant coefficients.

Performing one time step of the scheme~\ref{eq:app-roe-general-scheme} gives

\begin{align}
 \uc_i^{n+1}
    = \sum_\alpha A_\alpha \uc_{i+\alpha}^n
    =  \sum_\alpha A_\alpha \sum_{q = 0}^p \beta_q ((i + \alpha) \Delta x)^q
\end{align}

whereas the exact solution is given by

\begin{align}
\uc_{i,exact}^{n+1}
    = \uc_{i}^{n} (i \Delta x - a \Delta t, n \Delta t)
    = \sum_{q = 0}^p \beta_q (i \Delta x - a \Delta t)^q
\end{align}

Therefore the scheme gives the exact solution if and only if

\begin{align}
    &\uc_i^{n+1} = \uc_{i, exact}^{n+1} \\
    &\sum_\alpha A_\alpha \uc_{i+\alpha}^n =
    \sum_\alpha A_\alpha \sum_{q = 0}^p \beta_q ((i + \alpha) \Delta x)^q
    = \sum_{q = 0}^p \beta_q (i \Delta x - a \Delta t)^q
\end{align}

This equality must hold for each exponent $q$ individually, so we can look at each
summand with index $q$ individually:


\begin{align}
\sum_\alpha A_\alpha \beta_q ((i + \alpha) \Delta x)^q = \beta_q (i \Delta x - a \Delta t)^q
\end{align}

Dividing the equation by $\beta_q$ and $\Delta x^q$ gives


\begin{align}
\sum_\alpha A_\alpha (i + \alpha)^q = \left(i - \frac{a \Delta t}{\Delta x}\right)^q \ .
\end{align}

Since this must hold for all cell indices $i$, we can without loss of generality select $i = 0$ and
write


\begin{align}
\sum_\alpha A_\alpha \alpha^q = \left( - \frac{a \Delta t}{\Delta x}\right)^q = (- C_{CFL})^q
\end{align}

This relation must hold for each exponent $0 \leq q \leq p$ individually, and hence the first
theorem is proven.


It remains to demonstrate the second theorem, which states that a scheme of
form~\ref{eq:app-roe-general-scheme} that satisfies condition~\ref{eq:app-roe-theorem1} is $p$-th
order accurate.

A scheme is called $p$-th order accurate if the leading term of the point-wise error is proportional
to $\Delta x^p$. The point-wise error $P_i^{n+1}$ is defined as the difference between the exact
solution and the numerical solution divided by the time step:

\begin{align}
 P_i^{n+1} = \frac{\uc_{i,exact}^{n+1} - \uc_i^{n+1}}{\Delta t}
\end{align}

Without loss of generality, we once again consider the case for cell index $i = 0$. The point-wise
error is $P_0^{n+1}$ is then given by

\begin{align}
P_0^{n+1}
&= \frac{1}{\Delta t} \left[
    \uc_(i=0, (n+1) \Delta t) - \uc_{i=0}^{n+1}
    \right] \\
&= \frac{1}{\Delta t} \left[
    \uc(-a\Delta t, n \Delta t) - \sum_\alpha A_\alpha \uc_\alpha^n
    \right]
\end{align}

Now we Taylor-expand both the terms $\uc(-a \Delta t, n \Delta t)$ and $\uc_\alpha^n$ around $x =
0$ to obtain


\begin{align}
P_0^{n+1}
&= \frac{1}{\Delta t} \left[
    \sum_{q = 0}^\infty \frac{(-a \Delta t)^q}{q!} \frac{\del^q \uc^n}{\del x^q} -
    \sum_\alpha A_\alpha
        \left( \sum_{q=0}^\infty \frac{(\alpha \Delta x)^q}{q!} \frac{\del^q \uc^n}{\del x^q}
\right)
    \right] \\
&= \frac{1}{\Delta t} \sum_{q = 0}^\infty  \frac{1}{q!} \left[
    (-a \Delta t)^q - \Delta x^q \sum_\alpha A_\alpha \alpha^q
    \right] \frac{\del^q \uc^n}{\del x^q}  \\
&= \frac{1}{\Delta t} \sum_{q = 0}^\infty  \frac{\Delta x^q}{q!} \left[
    \left(\frac{-a \Delta t}{\Delta x} \right)^q - \sum_\alpha A_\alpha \alpha^q
    \right] \frac{\del^q \uc^n}{\del x^q} \\
&= \frac{1}{\Delta t} \sum_{q = 0}^\infty  \frac{\Delta x^q}{q!} \left[
    \left(- C_{CFL} \right)^q - \sum_\alpha A_\alpha \alpha^q
    \right] \frac{\del^q \uc^n}{\del x^q}
\end{align}

Using the property~\ref{eq:app-roe-theorem1}, it is obvious that the term in parentheses vanishes
for all $q \leq p$, and hence the leading term of the error is given by $q = p +1$:

\begin{align}
\frac{1}{\Delta t} \frac{\Delta x^{p + 1}}{(p + 1)!} \left[
    \left(- C_{CFL} \right)^{p + 1} - \sum_\alpha A_\alpha \alpha^{p + 1}
    \right] \frac{\del^{p + 1} \uc^n}{\del x^{p + 1}} \\
=
\frac{a \Delta x^p}{C_{CFL} (p + 1)!} \left[
    \left(- C_{CFL} \right)^{p + 1} - \sum_\alpha A_\alpha \alpha^{p + 1}
    \right] \frac{\del^{p + 1} \uc^n}{\del x^{p + 1}}
\end{align}


The leading term is proportional to $\Delta x^p$, and therefore a method that satisfies
condition~\ref{eq:app-roe-theorem1} will be $p$-th order accurate.
