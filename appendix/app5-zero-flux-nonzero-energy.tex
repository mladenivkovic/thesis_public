%---------------------------------------------------------------------------------
\chapter{A Time Step For A Particle With Nonzero Radiation Energy Density And Zero Radiation Flux}
\label{app:zero-flux-nonzero-energy}
%---------------------------------------------------------------------------------

We could encounter cases where we have nonzero radiation energy, but zero radiation flux, for
example through diffusion, or when exception handling unphysical scenarios, or by intentionally only
injecting energy density.) In these cases, recall that (index $i$ below is for photon
frequency group, not particle index!)

\begin{align}
	\DELDT{\Fbf_i} + c^2 \ \nabla \cdot \mathds{P}_i &=
		- \sum\limits_{j}^{\mathrm{HI, HeI, HeII}} n_j \sigma_{i j} c \Fbf_i \\
	\mathds{P}_i &=
		\mathds{D}_i E_i \\
	\mathds{D}_i &=
		\frac{1- \chi_i}{2} \mathds{I} + \frac{3 \chi_i - 1}{2} \mathbf{n}_i \otimes \mathbf{n}_i\\
	\mathbf{n}_i &=
		\frac{\Fbf_i}{|\Fbf_i|} \\
	\chi_i &=
		\frac{3 + 4 f_i ^2}{5 + 2 \sqrt{4 - 3 f_i^2}} \\
	f_i &=
		\frac{|\Fbf_i|}{c E_i}
\end{align}

For $\Fbf_i = 0$, we get
\begin{align}
	f_i &= 0 \\
	\chi_i &= \frac{3}{5 + 2 \sqrt{4}} = \frac{1}{3} \\
	\mathds{D}_i &= \frac{1- \frac{1}{3}}{2} \mathds{I} = \frac{1}{3} \mathds{I} \\
	\mathds{P}_i &= \mathds{D}_i E_i =  \frac{1}{3} E_i \mathds{I}
\end{align}

which is the solution of the optically thick limit, where the radiation pressure tensor is
isotropic.
Now let's look at what happens when a particle $k$ has some nonzero energy $E_0$, but zero flux,
and interacts with particle $l$, which has both zero energy density and zero flux. The initial
state is then:

\begin{align}
    \mathcal{U}_k(t = 0) &= \left( \begin{matrix}
                    E_0 \\
                    \mathbf{0}
                  \end{matrix} \right)
    \quad & \quad \quad
    \F_k(t = 0)  &= \left( \begin{matrix}
                   \mathbf{0} \\
                   c^2 \mathds{P}
                  \end{matrix} \right)
                = \left( \begin{matrix}
                   \mathbf{0} \\
                   \frac{c^2}{3} E_0 \mathds{I}
                  \end{matrix} \right)\\
%
    \mathcal{U}_l(t = 0) &= \left( \begin{matrix}
                    0 \\
                    \mathbf{0}
                  \end{matrix} \right)
    \quad & \quad \quad
    \F_l(t = 0)  &= \left( \begin{matrix}
                   \mathbf{0} \\
                   0
                  \end{matrix} \right)
\end{align}

To simplify matters, let's assume both particles have equal volumes, $V_k = V_l = V$, and let's
omit the gradient extrapolation to the interface position that makes the method second order
accurate. Let's also assume that the particles are aligned parallel to a coordinate axis,  thus
making the projection along the normal vector to their interaction surface trivial.

Then the intercell flux given by the GLF Riemann solver (eq. \ref{eq:riemann-GLF}) is

\begin{align}
	\F_{1/2}(\mathcal{U}_L, \mathcal{U}_R) &=
		\frac{\F_{L} + \F_{R}}{2} -
		\frac{c}{2} \left(\mathcal{U}_R - \mathcal{U}_L \right) \\
	&=	\left(\begin{matrix}
        \frac{0 - 0}{2} - \frac{c}{2} (0 - E_0)\\
        \frac{c^2 / 3 \ E_0 - 0}{2} - \frac{c}{2} (0 -0)
	  	\end{matrix} \right)
	=	\left(\begin{matrix}
        \frac{c}{2} E_0 \\
        \frac{c^2}{6} E_0
	  	\end{matrix} \right)
\end{align}

Let $\beta \equiv \frac{\Delta t \mathcal{A}_{kl} c}{2 V}$. Using the update formula given in eq.
\ref{eq:transport-update-explicit}, the states at $t = \Delta t$ will be

\begin{align}
	\mathcal{U}_k(t = \Delta t) &= \mathcal{U}_k (t = 0) - \frac{\Delta t}{V} \F_{1/2}
\mathcal{A}_{kl} \\
	&=	\left(\begin{matrix}
        E_0 -  \frac{\Delta t \mathcal{A}_{kl}}{V} \frac{c}{2} E_0 \\
        \mathbf{0} - \frac{\Delta t \mathcal{A}_{kl}}{V} \frac{c^2}{6} E_0
	  	\end{matrix} \right)
    = \left( \begin{matrix}
             E_0 (1 - \beta) \\
             -\frac{\beta}{3} c E_0
             \end{matrix} \right) \\
%
	\mathcal{U}_l(t = \Delta t) &= \mathcal{U}_l (t = 0) - \frac{\Delta t}{V} (-F_{1/2})
\mathcal{A}_{kl} \label{eq:zero-flux-update-l}\\
	&=	\left(\begin{matrix}
        0 +  \frac{\Delta t \mathcal{A}_{kl}}{V} \frac{c}{2} E_0 \\
        \mathbf{0} + \frac{\Delta t \mathcal{A}_{kl}}{V} \frac{c^2}{6} E_0
	  	\end{matrix} \right)
    = \left( \begin{matrix}
             \beta E_0 \\
             \frac{\beta}{3} c E_0
             \end{matrix} \right)
\end{align}

where the minus sign for the inter-cell flux $(-\F_{1/2})$ in eq. \ref{eq:zero-flux-update-l} stems
from the orientation of the effective surface $\mathcal{A}_{kl}$.


The takeaway here is that after a single time step, both particles $k$ and $l$ will have nonzero
energy \emph{and} nonzero photon flux $\Fbf$. For the photon fluxes of the particle $l$, we see that
$|\Fbf_l| = \frac{1}{3} c E_l$ with $E_l = \beta E_0$, meaning that the photon flux corresponds to
the optically thick, diffusion limit. For particle $k$ we have $|\Fbf_k| = \frac{\beta}{1 - \beta} c
E_k$ with $E_k = (1 - \beta) E_0$. Comparing the particle volume $V$ and effective surface
$\mathcal{A}_{kl}$ to a regular cell of edge length $L$, then $V = L^3$ and $\mathcal{A}_{kl} =
L^2$. Simultaneously this gives us a comparative CFL condition: $L \geq c \Delta t$. This simple
comparison limits the possible values of $\beta$ to $0 \leq \beta \leq \frac{1}{2}$, given that all
terms of $\beta$ must be positive. The limits translate to $0 \leq |\Fbf_k| \leq c E_k$ for the
photon flux, meaning that particle $k$ can end up in any state between no net flux, and the free
streaming, optically thin limit.


