\cleardoublepage
\chapter*{Abstract}
\markboth{Abstract}{Abstract}
\addcontentsline{toc}{chapter}{Abstract}

Numerical simulations have become an indispensable tool in astrophysics and cosmology. The constant
need for higher accuracy, higher resolutions, and models of ever-increasing sophistication and
complexity drives the development of modern tools which target largest computing systems and employ
state-of-the-art numerical methods and algorithms. Hence modern tools need to be developed while
keeping optimization and parallelization strategies in mind from the start.

In this work, the development and implementation of \GEARRT, a radiative transfer solver using the
M1 closure in the open source code \swift, is presented, and validated using standard tests for
radiative transfer. \GEARRT is modeled after \codename{Ramses-RT} (\cite{ramses-rt13}) with some key
differences. Firstly, while \codename{Ramses-RT} uses finite volume methods and an adaptive mesh
refinement (AMR) strategy, \GEARRT employs particles as discretization elements and solves the
equations using a Finite Volume Particle Method (FVPM).
% While in its current state \GEARRT requires hydrodynamics to also be solved using a FVPM, it can be
% extended to work alongside hydrodynamics solved using a smooth particle hydrodynamics (SPH) method
% as well.
Secondly, \GEARRT makes use of the task-based parallelization strategy of \swift, which allows for
optimized load balancing, increased cache efficiency, asynchronous communications, and a domain
decomposition based on work rather than on data.

\GEARRT is able to perform sub-cycles of radiative transfer steps w.r.t. a single hydrodynamics
step. Radiation requires much smaller time step sizes than hydrodynamics, and sub-cycling permits
calculations which are not strictly necessary to be skipped. Indeed, in a test case with gravity,
hydrodynamics, and radiative transfer, the sub-cycling is able to reduce the runtime of a simulation
by over 90\%. Allowing only a part of the involved physics to be sub-cycled is a contrived matter
when task-based parallelism is involved, and it required the development of a secondary time
stepping scheme parallel to the one employed for other physics. It is an entirely novel feature
in \swift.

Since \GEARRT uses a FVPM, a detailed introduction into finite volume methods and finite volume
particle methods is presented. In astrophysical literature, two FVPM methods are written about:
\cite{hopkinsGIZMONewClass2015} have implemented one in their \codename{Gizmo} code, while the one
mentioned in \cite{ivanovaCommonEnvelopeEvolution2013} isn't used to date. In this work, I test an
implementation of the \cite{ivanovaCommonEnvelopeEvolution2013} version, and conclude that in its
current form, it is not suitable for use with particles which are co-moving with the fluid, which
in turn is an essential feature for cosmological simulations.

Finally, the implementation of \acacia, a new algorithm to generate dark matter halo merger trees
with the AMR code \ramses, is presented.
% The algorithm is fully parallel and based on the Message Passing Interface (MPI).
As opposed to most available merger tree tools, it works on the fly during the course of the
$N$-body simulation. It can track dark matter substructures individually using the index of the most
bound particles in the clump.
Once a halo (or a sub-halo) merges into another one, the algorithm still tracks it through the last
identified most bound particle in the clump, allowing to check at later snapshots whether the
merging event was definitive. %, or whether it was only temporary, with the clump only traversing
another one.
%The same technique can be used to track orphan galaxies that are not assigned to a
% parent clump anymore because the clump dissolved due to numerical over-merging.
The performance of the method is compared using standard validation diagnostics, demonstrating that
it reaches a quality similar to the best available and commonly used merger tree tools. As proof of
concept, \acacia is used together with a parameterized stellar-mass-to-halo-mass relation to generate
a mock galaxy catalog that shows good agreement with observational data.

% The performance of the method using standard
% We study in detail the impact of various parameters on the resulting halo catalogues and
% corresponding merger histories.
\hspace{1cm}

\textbf{Keywords}: Numerical Methods -- Numerical Simulations -- Finite Volume Methods -- Finite
Volume Particle Methods -- Meshless Methods -- Radiative Transfer -- Parallel Computing -- Galaxies
Formation -- Galaxies Evolution -- Merger Trees -- Dark Matter



\cleardoublepage

\begin{otherlanguage}{german}
\chapter*{Zusammenfassung}
\markboth{Zusammenfassung}{Zusammenfassung}

Numerische Simulationen sind zu einem unabdiglichen Werkzeug in Astrophysik und Kosmologie
geworden. Der best\"andige Bedarf nach h\"oherer Genauigkeit, nach h\"oherer Aufl\"osung, und nach
Modellen mit stetig steigender Rafinesse und Komplexit\"at treibt die Entwicklung von modernen
Werkzeugen an, welche die gr\"ossten Rechnersysteme anzielen und von numerischen Methoden und
Algorithmen des neusten Standes Gebrauch machen. Statt sich nur auf die Physik und die Modelle zu
konzentrieren, m\"ussen moderne Werkzeuge von Beginn an unter Beachtung von Optimisation und
Parallelisationsstrategien entwickelt werden.

In dieser Arbeit wird die Entwicklung und Implementation von \GEARRT pr\"asentiert, das
Strahlungs\"ubertragung mittels der M1 Schliessung l\"ost, und durch Standardtests f\"ur
Strahlungs\"ubertragung validiert. \GEARRT ist nach dem Vorbild von \codename{Ramses-RT}
(\cite{ramses-rt13}) geformt, jedoch mit einigen entscheidenden Unterschieden. Erstents benutzt
\codename{Ramses-RT} Finite-Volumen-Verfahren und eine adaptive Gitterverfeinerungsstrategie
(Adaptive Mesh Refinement, AMR), w\"ahrend \GEARRT Teilchen als Diskretisationselemente einsetzt
und die Gleichungen mittels eines Finite-Volumen-Teilchen-Verfahrens (FVTV) l\"ost. Zweitens
benutzt \GEARRT die aufgabenbasierte Parallelisationsstrategie von \swift, welche eine optimisierte
Lastbalancierung, verbesserte Cache-Effizienz, asynchrone Kommunikationen, und eine arbeitsbasierte
Dom\"anendekomposition statt einer datenbasierten erlaubt.

\GEARRT ist f\"ahig Unterzyklen der Strahlungs\"ubertragung im Bezug auf einzelne hydrodynamische
Schritte zu t\"atigen. Strahlung ben\"otigt viel kleinere Zeitschritte als Hydrodynamik, und der
Einsatz von Unterzyklen erlaubt es Schritte, welche nicht streng n\"otig sind, zu \"uberspringen.
In einem Versuch mit Gravitation, Hydrodynamik, und Strahlungs\"ubertragung war der Einsatz von
Unterzyklen imstande, die Laufzeit der Simulation \"uber 90\% zu reduzieren.
Nur einen Teil der beteiligten Physik in Unterzyklen zu l\"osen ist eine vertrackte Angelegenheit
unter Anbetracht der aufgabenbasierten Parallelisationsstrategie, und erforderte die Entwicklung
einer sekund\"aren Zeitschrittmethode, die parallel zu der anderen (welche die restliche Physik
l\"ost) l\"auft. Der Einsatz von Unterzyklen ist eine vollst\"andig neue Funkion in \swift.

Da \GEARRT ein FVTV einsetzt, ist eine detaillierte Einf\"uhrung in Finite-Volumen- und
Finite-Volumen-Teilchen-Verfahren gegeben. In der Literatur der Astrophysik wird \"uber zwei FVTV
geschrieben: \cite{hopkinsGIZMONewClass2015} haben eine in ihrem \codename{Gizmo} Code implementiert,
w\"ahrend diejenige, welche in \cite{ivanovaCommonEnvelopeEvolution2013} erw\"ahnt wird, bis heute
nicht benutzt wird. In dieser Arbeit teste ich eine Implementation der
\cite{ivanovaCommonEnvelopeEvolution2013}-Version, und komme zum Schluss, dass es in der heutigen
Form unbrauchbar f\"ur den Einsatz mit Teilchen ist, welche mit dem Fluid mitbewegt werden, was
von essentieller Wichtigkeit in kosmologischen Simulationen ist.


Letztendlich wird die Implementation von \acacia, einem neuen Algorithmus um Verschmelzungsb\"aume
(Merger Trees) von Halos bestehend aus dunkler Materie, im AMR Code \ramses pr\"asentiert. Im
Unterschied zu den meisten verf\"ugbaren Merger Tree Werkzeugen arbeitet \acacia noch w\"ahrend die
$N$-Teilchen-Simulation verl\"auft. Es kann Unterstrukturen von dunkler Materie durch die Indizes
von den am st\"arksten gebundenen Teilchen der Klumpen einzeln verfolgen. Wenn ein Halo (oder
Unter-Halo) mit einem anderen verschmilzt, wird es vom Algorithmus durch den zuletzt als am
st\"arksten gebunden identifizierten Teilchen weiterhin verfolgt. Dies erm\"oglicht es in sp\"ateren
Speicherabz\"ugen zu pr\"ufen, ob die Verschmelzung tats\"achlich definitiv war. Die Leistung der
Methode wird mittels Standardvalidationsdiagnostiken verglichen, und zeigt, dass es vergleichbare
Qualit\"at mit den besten verf\"ugbaren und h\"aufig benutzten Merger Tree Werkzeugen liefert. Als
Konzeptnachweis wird \acacia zusammen mit einer parametrisierten Stern-Masse-Zu-Halo-Masse-Beziehung
eingesetzt, um k\"unstliche Galaxienkataloge zu erstellen, welche gute \"Ubereinstimmung mit
observierten Daten aufweisen.

\end{otherlanguage}
