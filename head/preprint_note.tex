\cleardoublepage

\begin{small}

%---------------------------------------
\chapter*{Note on this Preprint}
%---------------------------------------

This is the preprint version of my PhD Thesis submitted to the \'Ecole Polytechnique F\'ed\'erale 
de Lausanne in February 2023, with some very minor changes to the printed version, mostly in order 
to remove personal data floating around the wilderness of the internet.

For quick access, here is a list of links to codes and software libraries used throughout this work.


%---------------------------------------
\subsection*{This Work}
%---------------------------------------

This work is available on \url{https://github.com/mladenivkovic/thesis_public}.


%---------------------------------------
\subsection*{Finite Volume Code}
%---------------------------------------

\meshhydro (mentioned and used in Part~\ref{part:finite-volume}) is publicly available under
\url{https://github.com/mladenivkovic/mesh-hydro}



%-------------------------------------------------
\subsection*{Finite Volume Particle Methods}
%-------------------------------------------------

The python module to visualise ``effective surfaces'' discussed in Part~\ref{part:meshless} is 
available on \href{https://github.com/mladenivkovic/astro-meshless-surfaces}{PyPI.org} and on 
\url{https://github.com/mladenivkovic/astro-meshless-surfaces}.


%-------------------------------------------------
\subsection*{\GEARRT and \swift}
%-------------------------------------------------

\GEARRT is part of \swift. \swift is available under \url{https://github.com/swiftsim/swiftsim} and 
\url{https://gitlab.cosma.dur.ac.uk/swift/swiftsim}.

Online documentation for \swift can be found on \url{swiftsim.com/docs}, and is also shipped along 
with the git repository \swift comes in. An onboarding guide for a quick start with \swift is 
provided under \url{swiftsim.com/onboarding.pdf} (and is also part of the \swift git 
repository).\\[1em]

Additional radiative transfer related tools and tests for \GEARRT and \swift, including all the 
tests following the \citet{ilievCosmologicalRadiativeTransfer2006} and 
\citet{ilievCosmologicalRadiativeTransfer2009} comparison project presented in 
Sections~\ref{chap:IL6} and \ref{chap:IL9} can be found in 
\url{https://github.com/SWIFTSIM/swiftsim-rt-tools}.\\[1em]

This work made heavy use of the \codename{swiftsimio} \citep{borrowSwiftsimioPythonLibrary2020} 
visualisation and analysis python library. It is available on 
\href{https://pypi.org/project/swiftsimio/}{PyPI.org} and on 
\url{https://github.com/SWIFTSIM/swiftsimio}. Documentation is available under 
\url{https://swiftsimio.readthedocs.io}.




%-------------------------------------------------
\subsection*{\acacia}
%-------------------------------------------------

\acacia is part of \ramses, and available under \url{https://bitbucket.org/rteyssie/ramses/}.

\end{small}
